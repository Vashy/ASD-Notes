\section{Raccolta algoritmi}

\subsection{Insertion Sort}
Per approfondire, vedi la sezione \ref{insertionsort} 
\begin{codebox}
\Procname{$\proc{Insertion-Sort}(A)$}
\li $n \gets \attrib{A}{length}$
\li \For $j \gets 2$ \To $n$
	\Comment il primo elemento è già ordinato
\li		\Do
			$\id{key} \gets A[j]$ 
			\Comment $A[1 \twodots j-1]$ ordinato
\li			$i \gets j-1$
\li			\While $i > 0$ and $A[i] > \id{key}$
\li				\Do
					$A[i+1] \gets A[i]$
\li					$i \gets i-1$
				\End
\li			$A[i+1] \gets \id{key}$
		\End
\end{codebox}

\subsection{Merge Sort}
Vedi la sezione \ref{mergesort}
\begin{codebox}
\Procname{$\proc{Merge-Sort}(A,p,r)$}
\li \If $p < r$
\li     \Then
            $q \gets (p + r) / 2$ 
        \Comment arrotondato per difetto
\li         $\proc{Merge-Sort}(A,p,q)$
        \Comment ordina \texttt{A[p$\twodots$q]}
\li         $\proc{Merge-Sort}(A,q+1,r)$
        \Comment ordina \texttt{A[q+1$\twodots$r]}
\li         $\proc{Merge}(A,p,q,r)$
        \Comment ``Merge'' dei due sotto-array 
        \End
\end{codebox}
\input{pseudocodes/merge.tex}

\subsection{Insertion Sort ricorsivo}

\begin{codebox}
\Procname{$\proc{Insertion-Sort}(A,j)$}
\li \If $j > 1$
\li     \Then
		$\proc{Insertion-Sort}(A,j-1)$
        \Comment ordina \texttt{A[1$\twodots$j-1]}
\li         $\proc{Insert}(A,j)$
        \Comment inserisce \texttt{A[j]} in modo ordinato in \texttt{A}
        \End
\end{codebox}
\begin{codebox}
\Procname{$\proc{Insert}(A,j)$}
\zi	\Comment Precondizione: \texttt{A[1$\twodots$j-1]} è ordinato
\li	\If $(j > 1) \ \kw{and} \ (A[j] < A[j-1])$
\li		\Then
			$A[j] \leftrightarrow A[j-1]$
		\Comment scambia le celle \texttt{j} e \texttt{j-1}
\zi		\Comment se le celle sono state scambiate, ordina 
\zi		\Comment il nuovo sottoarray \texttt{A[1$\twodots$j-1]}
\li			$\proc{Insert}(A,j-1)$
		\End
\end{codebox}

\subsubsection{Correttezza di Insertion-Sort(A,j)}
Procediamo per induzione:
\begin{itemize}
	\item[] $(j \leq 1) \quad$ Caso base. Array già ordinato, non faccio nulla $\Rightarrow$ ok;
	\item[] $(j > 1) \quad$ Per ipotesi induttiva, la chiamata \texttt{Insertion-Sort(A,j-1)}
	ordina \texttt{A[1$\twodots$j-1]}. Assumendo la correttezza di \texttt{Insert(A,j-1)}, esso
	``inserisce'' \texttt{A[j]} $\Rightarrow$ produce \texttt{A[1$\twodots$j]} ordinato.
\end{itemize}

\subsubsection{Correttezza di Insert(A,j)}
Anche qui, dimostrazione per induzione:
\begin{itemize}
	\item[] $(j = 1) \quad$ Caso base. \texttt{A[1]} da inserire nell'array vuoto. Non fa nulla
	$\Rightarrow$ ok;
	\item[] $(j > 1) \quad$ Due sottocasi:
	\begin{itemize}
		\item \texttt{A[j]} $\geq$ \texttt{A[j-1]}: non faccio nulla, \texttt{A[1$\twodots$j]} già
		ordinato;
		\item \texttt{A[j]} $<$ \texttt{A[j-1]}: scambio le chiavi delle due celle. Il nuovo \texttt{A[j]}
		sarà sicuramente maggiore di qualsiasi altro elemento che lo precede, poiché, per precondizione di 
		\texttt{Insert}, \texttt{A[1$\twodots$j-1]} era ordinato, e dato che valeva \texttt{A[j-1]} $\geq$ 
		\texttt{A[j]}, il nuovo \texttt{A[j]} (che è il precedente \texttt{A[j-1]}) sarà sicuramente l'elemento con il valore più alto. 
		Dopodichè, chiamo \texttt{Insert(A,j-1)} per ordinare la cella \texttt{A[j-1]}.
 	\end{itemize}
\end{itemize}

\subsection{CheckDup}
Algoritmo che verifica la presenza di duplicati in \texttt{A[p$\twodots$r]} e, 
solo se non ci sono, ordina l'array.
\begin{codebox}
\Procname{$\proc{Check-Dup}(A,p,r)$}
\li \If $p < r$
\li     \Then
            $q \gets \floor{\frac{p + r}{2}}$ 
        \Comment arrotondato per difetto
\li         \Return $\proc{Check-Dup}(A,p,q)$
\li         \mbox{ or } $\proc{Check-Dup}(A,q+1,r)$
\li         \mbox{ or } $\proc{DMerge}(A,p,q,r)$
        \End
\end{codebox}