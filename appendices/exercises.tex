\section{Esercizi}
\subsection{Ricorrenze}

\begin{itemize}[label=$\bullet$]
	\item $T(n) = aT(n-1) + b \qquad a, b > 1$
	\begin{itemize}
		\item \emph{radice}: costo $b$;
		\item la radice ha $a$ figli di costo $b$;
		\item \dots
		\item foglie terminali $O(1)$.
	\end{itemize}
	Esplicitando il caso base della ricorrenza otteniamo:
	\[ 
		T(n) =
		\begin{cases}
			c & n = 0 \\
			aT(n-1) + b & n > 0
		\end{cases}
	\]
	\begin{align*}
		T(n) & = b + ab + a^2b + \dots + a^{n-1}b + a^nc \\
		& = b \displaystyle\sum_{j=0}^{n-1}a^j + a^nc \hspace{1cm} \text{(dimostrare per induzione)}
	\end{align*}
	\begin{description}
		\item[$(a=1)$] $T(n) = nb + c = \Theta(n)$
		\item[$(a<1)$] $T(n) = \frac{1-a^n}{1-a} \cdot b + a^n c = \Theta(1)$ \par
		$\text{(valgono } \frac{1-a^n}{1-a} \leq \frac{1}{1-a}, \; a^nc < c \text{)}$
		\item[$(a>1)$] $T(n) = \frac{a^n-1}{a-1}b + a^n c = \Theta(a^n)$
	\end{description}
\end{itemize}

\subsection{Esercizi svolti il 06/04/2018} \label{exs:6-4-2018}

\paragraph{Gap} Abbiamo un \emph{gap} se $i < n$ t.c. $A[i+1] - A[i] > 1$.

Mostrare per induzione che se $A[n] - A[1] \geq n$ allora c'è un gap.

\subparagraph{Sol.}
\begin{itemize}
	\item \underline{Base}
	\begin{itemize}
		\item $n = 1 \Rightarrow A[1] - A[1] = 0$;
		\item $n = 2 \Rightarrow A[2] - A[1] > 1$ allora 1 è un gap.
	\end{itemize}
	\item \underline{Passo induttivo} $n > 2$
		Se $A[n] - A[1] \geq n$ abbiamo due casi:
		\begin{enumerate}
			\item Se $A[n] - A[n-1] > 1$ allora $A[n-1]$ è un \emph{gap};
			\item Se $A[n] - A[n-1] \leq 1$ allora
			\begin{gather*}
				A[n-1] - A[1] = \left( A[n] - A[1] \right) - \left( A[n] - A[n-1] \right) \\
				(\text{valgono } A[n-1]-A[1] \geq n \text{ e } A[n]-A[n-1] \leq 1) \\
				A[n-1] - A[1] \geq n-1 \Rightarrow A[1\twodots n-1] \text{ contiene un \emph{gap}}
			\end{gather*}
		\end{enumerate}
\end{itemize}

\begin{codebox}
\Procname{\proc{Gap}$(A,p,r)$} 
\zi \Comment pre: $r - p > 1$
\li \If $p = r + 1$
\li     \Then \Return $p$
\li     \Else
\li         $q \gets \frac{p+r}{2}$
\li     \If $A[q] - A[p] > q - p + 1$
\li         \Then
                $\proc{Gap}(A,p,r)$
\li         \Else
\li             $\proc{Gap}(A,q+1,r)$
        \End
    \End
\end{codebox}

Valutiamo la complessità con il \emph{Master Theorem}.

$$T(n) = T\left( \frac{n}{2} \right) + \Theta(1)$$
Abbiamo $a = 1, \ b = 2$. Confronto $f(n) = \Theta(1)$ con $n^{\log_2 1} = 1$.\par
Siamo nel \emph{secondo caso} del Master Theorem, poichè le due funzioni hanno lo stesso
andamento $\Rightarrow \Theta (\log^{\log_2 1} \log n) = \Theta(\log n)$

\paragraph{Select} \texttt{Select(A, k)}, ritorna il $k$-esimo elemento 
dell'array \texttt{A} se fosse ordinato.

\subparagraph{Sol.} Ci sono diverse soluzioni al problema.

\begin{enumerate}[label=($\arabic*$)]
	\item Trasformo $A$ in un \emph{MinHeap}, ed estraggo il minimo $k$ volte.
	\begin{codebox}
	\Procname{\proc{Select}$(A,k)$}
	\li	$\proc{BuildMaxHeap}(A)$ \Comment $\Theta(n)$
	\li \For $i \gets 1$ \To $k$
	\li \Do
			$x \gets \proc{ExtractMin}(A)$ \Comment $k \cdot O(\log n)$
		\End
	\li \Return $x$
	\end{codebox}

	Complessità: $O(n + k \log n)$

	\item Soluzione alternativa.
	\begin{codebox}
		\Procname{\proc{Select}$(A,k)$}
		\li	$\proc{BuildMaxHeap}(A,k)$ \Comment $\Theta(k)$
		\li \For $i \gets k+1$ \To $n$
		\li \Do
				\If $A[i] < A[1]$ \Comment $O((n-k) \log k)$
		\li		\Then
					$A[i] \leftrightarrow A[1]$
		\li			$\proc{MaxHeapify}(A,1,k)$
				\End
			\End
		\li \Return $A[1]$
	\end{codebox}

	Complessità: $O(k + (n-k) \log k) \cong O(n \log k)$
\end{enumerate}

\paragraph{Inversioni} Quante \emph{inversioni} ci sono in $A$? Implementare un algoritmo
\emph{Divide ed Impera} con complessità $O(n \log n)$ 

\subparagraph{Def (inversione)} Una coppia di indici $i, j$ tali che $i < j$ è un \emph{inversione}
per l'array $A$ se e solo se $A[i] > A[j]$.

\subparagraph{Sol.} Possiamo distinguere tre casi:
\begin{itemize}
	\item $i,j \in [p,q]$;
	\item $i, j \in [q+1,r]$;
	\item $i \in [p,q], j \in [q+1,r]$.
\end{itemize}

La soluzione che adotteremo conterà le inversioni dei tre casi separatamente, e restituirà
la somma dei tre valori ottenuti.

\begin{multline*}
	\proc{\# Inv}(p,r) = \proc{\# Inv}(p,q) + \proc{\# Inv}(q+1,r) \\
	+ | \{ (i,j) : i \in [p,q], j \in [q+1,r], A[i] > A[j] \} |
\end{multline*}

\begin{codebox}
\Procname{\proc{Inv}$(A,p,r)$}
\zi \Comment restituisce $\verb|#| inv$ e ordina l'array $A$
\li \If $p < r$ 
\li \Then
		$q = \frac{p+r}{2}$
\li		\Return $\proc{Inv}(A,p,q) + \proc{Inv}(A,q+1,r) + \proc{MergeInv}(A,p,q,r)$
\li	\Else
\li		\Return $0$
	\End
\end{codebox}

\begin{enumerate}[label=($\alph*$)]
	\item $\proc{Inv}$ non cambia gli elementi in $A[p,q]$ e $A[q+1,r]$;
	\item Il numero di inversioni ``a cavallo'' non cambia;
	\item $i,j$ inversione $A[i] > A[j]$ \par
		$\Rightarrow (i',j) \quad i' \in [i,q]$ inversione. 
\end{enumerate}

\begin{codebox}
\Procname{\proc{MergeInv}$(A,p,q,r)$}
\li $n_1 \gets q - p + 1$
\li $n_2 \gets r - q$
\li $L[1,n_1] \gets A[p,q]$
\li $R[1,n_2] \gets A[q+1,r]$
\li $L[n_1+1] \gets R[n_2+1] \gets \infty$
\li $i \gets j \gets 1$
\li $\id{inv} \gets 0$
\li \For $k \gets p$ \To $r$
\li \Do
		\If $L[i] \leq R[j]$
\li		\Then
			$A[k] \gets L[i]$
\li 		$i \gets i + 1$
\li 	\Else
\li 		$A[k] \gets R[j]$
\li 		$j \gets j + 1$
\li 		$\id{inv} \gets \id{inv} + n_1 - i + 1$ 
		\End
	\End
\li \Return $\id{inv}$
\end{codebox}