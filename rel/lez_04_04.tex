\section{Lezione del 04/04/2018}

\subsection{Chaining}
Il \emph{Chaining} propone come soluzione quella di mettere sulla tabella liste dinamiche
di elementi, invece che singoli elementi, in modo che in caso si incorra in una cella
già occupata dopo un \emph{hashing}, l'elemento venga inserito in coda (o in testa) alla
lista.

\paragraph{Idea} \texttt{T[i]} $=$ lista elementi $x$ tali che $h(x.key) = i$

\begin{codebox}
\Procname{\proc{Insert}$(T,x)$}
\li	$\text{Inserisci $x$ nella lista } T[h(x.key)]$
	\Comment $O(1)$
\end{codebox}
\begin{codebox}
\Procname{\proc{Delete}$(T,x)$}
\li	$\text{Delete $x$ from } T[h(x.key)]$
	\Comment $O(1)$
\end{codebox}
\begin{codebox}
\Procname{\proc{Search}$(T,k)$}
\li	$\text{Cerco in } T[h(k)] \text{ un elemento}$
\zi $x \text{ con chiave } k$ 
	\Comment $O(n)$
\end{codebox}
\texttt{Search} ha una complessità di $O(n)$, e questo è inaccettabile.
\begin{gather*}
	n = \# \text{elementi inseriti} \\
	m = \text{dimensione di } T \\
	\alpha = \frac{n}{m} \quad \text{fattore di carico} \\
	\alpha \text{ può essere }<, \ = \text{ oppure } > \text{ di } 1
\end{gather*}

\subsubsection{Hashing uniforme semplice}
Ogni elemento di \emph{input} è "mandato" da $h$ con la stessa probabilità $\left( \frac{1}{m} \right)$
in una delle $m$ celle.

\paragraph{Caso medio} $\Theta(1 + \alpha)$, 1 è l'accesso alla tabella.\par
Consideriamo $n_j$ il $j$-esimo elemento della cella $n$.
$$E[n_j] = \displaystyle\sum_{i = 1}^{n} \frac{1}{m} \cdot 1 = \frac{n}{m} = \alpha$$