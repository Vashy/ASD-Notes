\section{Lezione del 02/03/2018}

\subsection{Modello dei costi}
\paragraph{Assunzione} Tutte le istruzioni richiedono un tempo \underline{costante}.

Rivediamo l'algoritmo:

\begin{codebox}
\Procname{$\proc{Insertion-Sort}(A)$}
\li $n \gets \attrib{A}{length}$
\li \For $j \gets 2$ \To $n$
	\Comment il primo elemento è già ordinato
\li		\Do
			$\id{key} \gets A[j]$ 
			\Comment $A[1 \twodots j-1]$ ordinato
\li			$i \gets j-1$
\li			\While $i > 0$ and $A[i] > \id{key}$
\li				\Do
					$A[i+1] \gets A[i]$
\li					$i \gets i-1$
				\End
\li			$A[i+1] \gets \id{key}$
		\End
\end{codebox}

Diamo il nome $c_0$ alla chiamata del metodo, \texttt{InsertionSort(A)};
A ogni riga numerata, diamo il nome $c_1, c_2,\twodots ,c_8$
\footnote{($c_1$ corrisponde alla riga 1, $c_2$ alla riga 2 e così via)}.\par
Vediamo il \emph{costo} di ogni istruzione:

\begin{itemize}
    \item[] $\boldsymbol{c_0} \rightarrow 1$
    \item[] $\boldsymbol{c_1} \rightarrow 1$
    \item[] $\boldsymbol{c_2} \rightarrow n$
    \item[] $\boldsymbol{c_3} \rightarrow (n-1)$
    \item[] $\boldsymbol{c_4} \rightarrow (n-1)$
    \item[] $\boldsymbol{c_5} \rightarrow \displaystyle\sum_{j=2}^{n} t_j+1$
    \item[] $\boldsymbol{c_6}, \boldsymbol{c_7} \rightarrow \displaystyle\sum_{j=2}^{n} t_j$
    \item[] $\boldsymbol{c_8} \rightarrow (n-1)$
\end{itemize}

\subsection{Complessità di IS} 
\begin{displaymath}
    T^{IS}(n) = c_0 + c_1 + c_2n + (c_3+c_4+c_8)(n-1)
    + c_5\displaystyle\sum_{j=2}^{n}(t_j+1) + (c_6+c_7)\displaystyle\sum_{j=2}^{n}t_j
\end{displaymath}

$\boldsymbol{t_j}$ dipende, oltre che da $n$, dall'istanza dell'array
che stiamo considerando.
È chiaro che questo calcolo non da indicazioni chiare sull'effettiva
complessità dell'algoritmo.\par

\bigskip
Andiamo ad analizzare i 3 possibili casi:

\begin{enumerate}[label=\emph{\alph*})]
    \item Caso migliore (\ref{is:casomigliore})
    \item Caso peggiore (\ref{is:casopeggiore})
    \item Caso medio (\ref{is:casomedio})
\end{enumerate}

\subsubsection{Caso migliore} \label{is:casomigliore}

$\rightarrow A$ ordinato $\Rightarrow t_j = 0 $ $\forall j$

\bigskip
La \textbf{complessità} diventa:
\begin{displaymath}
    T^{IS}_{min}(n) = c_0 + c_1 + c_2n + (c_3+c_4+c_5+c_8)(n-1) 
    = an+b \approx n
\end{displaymath}

Ossia, si comporta come $n$. Il \emph{caso migliore} \textbf{non}
è interessante, visto che è improbabile si presenti.

\subsubsection{Caso peggiore} \label{is:casopeggiore}

$\rightarrow A$ ordinato in senso inverso $\Rightarrow \forall j$ $t_j = j-1$ \par
\bigskip
La \textbf{complessità} diventa: 

\begin{displaymath}
    T^{IS}_{max}(n) = c_0 + c_1 + c_2n + (c_3+c_4+c_8)(n-1)
    + c_5\displaystyle\sum_{j=2}^{n}j + (c_6+c_7)
    \displaystyle\sum_{j=2}^{n}(j-1)
\end{displaymath}

Per valutare il costo di $\displaystyle\sum_{j=2}^{n}j$ e di 
$\displaystyle\sum_{j=2}^{n}(j-1)$, usiamo la \textbf{somma di Gauss}: \par

\begin{equation}
    \displaystyle\sum_{i=1}^{n}i = \frac{n(n+1)}{2}
\end{equation}
\newpage
Otteniamo:

\begin{displaymath}
    \displaystyle\sum_{j=2}^{n}j = \frac{n(n+1)}{2}-1
\end{displaymath}

\begin{displaymath}
    \displaystyle\sum_{j=2}^{n}(j-1) = \displaystyle\sum_{i=1}^{n}n = \frac{(n-1)n}{2}
\end{displaymath}

Per finire, ricalcoliamo $T^{IS}_{max}(n)$

\begin{displaymath}
    T^{IS}_{max}(n) = a'n^2+b'n+c' \approx n^2
\end{displaymath}

\subsubsection{Caso medio} \label{is:casomedio}
Il caso medio è \emph{difficile da calcolare}, e in una considerevole parte dei casi,
coincide con il caso peggiore.\par
Comunque, l'idea è la seguente:

\begin{align*}
    \frac{\displaystyle\sum_{\text{perm. di input}}T^{IS}(p)}{n!} \approx n^2 && 
        \text{posso pensare che } t_j \cong \frac{j-1}{2}
\end{align*}

\subsection{Divide et Impera}

Un algoritmo di sorting \emph{divide et impera} si può suddividere in 3 fasi:

\begin{description}
    \item[divide] divide il problema dato in sottoproblemi più piccoli;
    \item[impera] risolve i sottoproblemi:
    \begin{itemize}
        \item ricorsivamente;
        \item la soluzione è nota (e.g. array con un elemento);
    \end{itemize}
    \item[combina] compone le soluzioni dei sottoproblemi in una soluzione del 
        problema originale.
\end{description}

\subsection{Merge Sort}
\href{https://en.wikipedia.org/wiki/Merge_sort}{Merge Sort}\footnote{Si %
consiglia uno sguardo all'algoritmo da altre fonti, poichè presentarlo %
graficamente in \LaTeX, come è stato visto a lezione, è arduo.} è un 
esempio di algoritmo divide et impera. Andiamo ad analizzarlo.

\paragraph{Astrazione} Consideriamo il seguente array A.
\begin{center}
	\begin{tabular}{|l|l|l|l||l|l|l|l|}
		\hline
		5 & 2 & 4 & 7 & 1 & 2 & 3 & 6 \\
		\hline
	\end{tabular}
\end{center}

Lo divido a metà, ottenendo due parti separate.

\begin{center}
	\begin{tabular}{|l|l|l|l|}
		\hline
		5 & 2 & 4 & 7 \\
		\hline
	\end{tabular}
	\hspace{1cm}
	\begin{tabular}{|l|l|l|l|}
		\hline
		1 & 2 & 3 & 6 \\
		\hline
	\end{tabular}
\end{center}

Consideriamo il primo, ossia \texttt{A[1$\twodots$4]} (A originale). Divido anche questo a metà.

\begin{center}
	\begin{tabular}{|l|l|}
		\hline
		5 & 2 \\
		\hline
	\end{tabular}
	\hspace{1cm}
	\begin{tabular}{|l|l|}
		\hline
		4 & 7 \\
		\hline
	\end{tabular}
\end{center}

Divido nuovamente a metà, ottenendo:

\begin{center}
	\begin{tabular}{|l|}
		\hline
		5 \\
		\hline
	\end{tabular}
	\hspace{1cm}
	\begin{tabular}{|l|}
		\hline
		2 \\
		\hline
	\end{tabular}
\end{center}

5 e 2 sono due blocchi già ordinati. Scelgo il minore tra i due e lo metto in prima 
posizione, mentre l'altro in seconda posizione, ottenendo un blocco composto da 2 e 5.\par
Riprendo con il blocco composto da 4 e 7. Lo divido in due blocchi da un elemento. Faccio lo stesso procedimento 
fatto per 2 e 5: metto in prima posizione 4 e in seconda posizione 7. La situazione
è la seguente:

\begin{center}
	\begin{tabular}{|l|l|}
		\hline
		2 & 5 \\
		\hline
	\end{tabular}
	\hspace{1cm}
	\begin{tabular}{|l|l|}
		\hline
		4 & 7 \\
		\hline
	\end{tabular}
\end{center}

So che i blocchi ottenuti contengono elementi ordinati. Data questa assunzione, posso ragionare 
nel seguente modo: considero il primo elemento dei due blocchi (il 2 in questo caso) e lo metto 
in prima posizione. Ora considero il successivo elemento di quel blocco e lo stesso elemento del 
blocco che non è stato selezionato, e inserisco nell'array l'elemento minore. Continuo fino ad 
ottenere un blocco ordinato.

\begin{center}
	\begin{tabular}{|l|l|l|l|}
		\hline
		2 & 4 & 5 & 7 \\
		\hline
	\end{tabular}
\end{center}

Faccio lo stesso ragionamento con la parte di array originale \texttt{A[5$\twodots$8]}, ottenendo

\begin{center}
	\begin{tabular}{|l|l|l|l|}
		\hline
		2 & 4 & 5 & 7 \\
		\hline
	\end{tabular}
	\hspace{1cm}
	\begin{tabular}{|l|l|l|l|}
		\hline
		1 & 2 & 3 & 6 \\
		\hline
	\end{tabular}
\end{center}

A questo punto, i blocchi da 4 contengono elementi tra loro ordinati. Faccio lo stesso procedimento
usato per comporli, per ottenere l'array originale ordinato. Considero:
\begin{itemize}[noitemsep]
    \item \texttt{A[1$\twodots$4]}: indice $i = 1$ per scorrerlo;
    \item \texttt{A[5$\twodots$8]}: indice $j = 1$ per scorrerlo;
\end{itemize}

Valuto \texttt{A[i]} e \texttt{A[j]}. \par
\begin{itemize}[noitemsep]
    \item Se \texttt{A[i]} $\leq$ \texttt{A[j]}, inserisco \texttt{A[i]} e incremento \texttt{i}. \par
    \item Altrimenti, inserisco \texttt{A[j]} e incremento \texttt{j}.
    \item Itero finchè entrambi gli indici non sono out of bounds.
\end{itemize}

\paragraph{Pseudocodice} Segue lo pseudocodice del 
\texttt{Merge Sort}.

\begin{codebox}
\Procname{$\proc{Merge-Sort}(A,p,r)$}
\li \If $p < r$
\li     \Then
            $q \gets (p + r) / 2$ 
        \Comment arrotondato per difetto
\li         $\proc{Merge-Sort}(A,p,q)$
        \Comment ordina \texttt{A[p$\twodots$q]}
\li         $\proc{Merge-Sort}(A,q+1,r)$
        \Comment ordina \texttt{A[q+1$\twodots$r]}
\li         $\proc{Merge}(A,p,q,r)$
        \Comment ``Merge'' dei due sotto-array 
        \End
\end{codebox}

\input{pseudocodes/merge.tex}

\subsubsection{Invarianti e correttezza}

\textbf{L} e \textbf{R} contengono rispettivamente 
\texttt{A[p$\twodots$q]} e \texttt{A[q+1$\twodots$r]}. 
L'indice \texttt{k} scorre \texttt{A}. Il sotto-array \texttt{A[p$\twodots$k-1]}
è ordinato, e contiene \texttt{L[1$\twodots$i-1]} e \texttt{R[1$\twodots$j-1]}.

\begin{center}
    $A[p \twodots k-1] \leq L[i \twodots n1], R[j \twodots n2]$ \\
    $\Downarrow$ \\
    $A[p \twodots k-1] = A[p \twodots r+1-1] \implies A[p \twodots r] \text{ ordinato}$
\end{center}

\paragraph{Dimostrazione per induzione su r-p} 
\begin{itemize}
    \item[$\Rightarrow$] Se $r-p == 0 $ (oppure $-1$) abbiamo al
    più un elemento $\implies$ array già ordinato.
    \item[$\Rightarrow$] Se $r-p > 0$ per ipotesi induttiva:
    \begin{itemize}
        \item \texttt{Merge-sort(A,p,q)} ordina \texttt{A[p$\twodots$q]};
        \item \texttt{Merge-sort(A,q+1,r)} ordina \texttt{A[q+1$\twodots$r]}; \par
        Per correttezza di \texttt{Merge()}, dopo la sua chiamata ottengo 
        \texttt{A[p$\twodots$r]} ordinato.
    \end{itemize}
\end{itemize} 