\section{Lezione del 28/03}

\subsection{Quicksort a tre partizioni}
\texttt{Quicksort} con \texttt{RandomizedPartition} funziona bene ed 
evita, quasi in ogni circostanza, di imbattersi nel caso pessimo, ad 
eccezione di un caso particolare: se in \emph{input} viene dato un 
array con tutti gli elementi uguali, si ottiene il temuto caso pessimo $O(n^2)$.\par
Per ovviare al problema, è sufficiente partizionare Quicksort in tre
partizioni invece di due. Dato un \emph{pivot} $x$, partizioniamo $A$ nel
seguente modo:

\begin{center}
    \begin{tabular}{|l|l|l|}
        \hline 
        $< x$ & $= x$ & $> x$ \\
        \hline
    \end{tabular}
\end{center}

Durante l'algoritmo, la disposizione sarà questa:

\begin{center}
    \begin{tabular}{|l|l|l|l|}
        \hline 
        $< x$ & $= x$ & $\quad$ & $> x$ \\
        \hline
    \end{tabular}
\end{center}
(La cella vuota è la regione ancora da esplorare).

\begin{codebox}
\Procname{\proc{Tripartition}$(A,p,r)$}
\li $x \gets A[r]$
\li $i \gets p-1$
\li $k \gets p$
\li $j \gets r$
\li \While $k < j$
\li     \Do
            \If $A[k] < x$
\li             \Then
                    $i \gets  i + 1$
\li                 $A[i] \leftrightarrow A[k]$
\li                 $k \gets k + 1$
\li             \Else \If $A[k] > x$
\li                 $j \gets j - 1$
\li                 $A[j] \Leftrightarrow A[k]$
\li             \Else
\li                 $k \gets k + 1$
                \End
        \End
\zi \Comment $k = j$
\li $A[j] \Leftrightarrow A[r]$
\li \Return $(i+1,j)$ \Comment restituisce una coppia di valori
\end{codebox} 

\begin{codebox}
    \Procname{\proc{Quicksort}$(A,p,r)$}
    \li \If $p < r$
    \li 	\Then
                $\id{q1},\id{q2} \gets \proc{Tripartition}(A,p,r)$
    \li 		$\proc{Quicksort}(A,p,\id{q1}-1)$
    \li 		$\proc{Quicksort}(A,\id{q2}+1,r)$
            \End
\end{codebox} 

\subsection{Limite inferiore}

\begin{itemize}[label=]
    \item \texttt{Input}: $a_1 \dots a_n$
    \item \texttt{Output}: permutazione $a'_1 \dots a'_n$ tale che
    $$a'_1 \leq a'_2 \leq \dots \leq a'_n$$
\end{itemize}

\paragraph{Confronti e assegnamenti} 

Osservazioni:
\begin{itemize}[label=$\rightarrow$]
    \item Se ``conto'' solo alcune operazioni il limite inferiore
    vale in generale. Consideriamo solo l'operatore di confronto;
    \item Elementi tutti distinti $(a_i \neq a_j \text{ se } i \neq j)$,
    l'operatore di confronto $==$ restituisce sempre $\const{False}$.
\end{itemize}

\subsubsection{Albero di Decisioni}

È una rappresentazione ``astratta'' delle possibili esecuzioni di un 
algoritmo di ordinamento su un input di dimensione fissata $A[1 \dots n]$.

\begin{itemize}[label=$\rightarrow$]
    \item nodi interni: 
    $$i \ : \ j \Rightarrow \text{confronta } A[i] \leq A[j]$$
    \item \emph{foglie} (ogni foglia è una possibile permutazione)
\end{itemize}

Rivediamo una versione di \texttt{Insertion Sort} basato su scambi.
\begin{codebox}
    \Procname{$\proc{Insertion-Sort}(A)$}
    \li $n \gets \attrib{A}{length}$
    \li \For $j \gets 2$ \To $n$
        \Comment il primo elemento è già ordinato
    \li		\Do
    			$i \gets j-1$
    \li			\While $i > 1$ and $A[i] > A[i+1]$
    \li				\Do
                        $A[i] \leftrightarrow A[i+1]$
    \li					$i \gets i-1$
                    \End
            \End
    \end{codebox}
\clearpage
Ecco un esempio di \emph{Albero delle Decisioni} per l'array $A[a_1,a_2,a_3]$ 
con 
$$a_1 = 1, \; a_2 = 2, \; a_3 = 3$$
\begin{figure}[!hb] 
    \centering
    \includegraphics[width=\textwidth]{img/decision-tree.png}
    \caption{Albero delle decisioni per l'array $A[1,2,3]$}
\end{figure}

\paragraph{Osservazione} 
$$ \text{Altezza dell'albero di decisione} = \text{limite inferiore per caso pessimo}$$
\begin{align*}
    \text{per }& \text{IS} \ \quad n^2 \\
    \text{per }& \text{MS} \quad n \log n
\end{align*}

\emph{In generale}, le foglie contengono \underline{tutte} le permutazioni.
$$\# foglie \geq n! \qquad (\# foglie \leq 2^h)$$
\begin{align*}
    h & \geq \log_2 n! \\
    & \geq \log_2 \left( n(n-1)(n-2) \dots \frac{n}{2}\right) \\
    & \geq \log_2 \left( \frac{n}{2}\left(\frac{n}{2}-1\right)\left(\frac{n}{2}-2\right) \dots \frac{n}{2}\right) = \\
    & = \log_2 \left( \frac{n}{2} \right)^{(\frac{n}{2})} 
        = \frac{n}{2} \left( \log_2 n - \log_22\right)
        = \frac{n}{2} (\log_2 n - 1) = \Theta (n \log n)
\end{align*}

\subsection{Ordinamento in tempo lineare}
Esistono degli algoritmi di ordinamento che, in certe condizioni e per certi input, permettono
di ordinare in tempo lineare $\Omega (n)$

\subsubsection{Counting Sort}
Assumo
\begin{itemize}[label=$-$]
    \item interi;
    \item in $[0,k]$
\end{itemize}

\begin{itemize}[label=]
    \item \texttt{Input}: $A[1\twodots n]$ con $A[j] \in [0,k] \forall j$;
    \item \texttt{Output}: $B[1\twodots n]$ permutazione ordinata di $A$;
    \item \texttt{Supporto}: $C[0\twodots k]$.
\end{itemize}

\begin{codebox}
\Procname{\proc{CountingSort}$(A,B,k)$}
\li $C[0\twodots k] \leftarrow 0$
\li \For $j \gets 1$ \To $\attrib{A}{length}$
\zi     \Then
        \Comment $C[x] = \# elem$ in $A$ con valore $x$
\li         $C[A[j]] \gets C[A[j]] + 1$ 
        \End
\li \For $i \gets 1$ \To $k$
\zi     \Then
        \Comment $C[x] = \# elem$ in $A$ con valore $\leq x$
\li            $C[i] \gets C[i-1] + C[i]$ 
        \End
\li \For $j \gets \attrib{A}{length}$ \To $1$
\li     \Then
            $B[C[A[j]]] \gets A[j]$
\li         $C[A[j]] \gets C[A[j]] - 1$
        \End
\end{codebox} 

\paragraph{Costo?} 
\begin{align*}
    C[0,k] \leftarrow 0 && \Theta(k) \\
    \text{\texttt{for j=1}} \dots && \Theta(n) \\
    \text{\texttt{for i=1}} \dots && \Theta(k) \\
    \text{\texttt{for j=A.length}} \dots && \Theta(n)
\end{align*}
$$\text{Somma } \Theta(n+k) \text{ con } k = \Theta(1) \Rightarrow \Theta(n)$$

\paragraph{Problema di memoria} Il problema di \texttt{Counting Sort} è la memoria. 
Infatti, al crescere di $k$, la memoria richiesta per allocare \texttt{C} cresce esponenzialmente.
\begin{center}
    \begin{tabular}{|l|l|}
        \hline
        Dimensione $k$ & Memoria occupata da \texttt{C[]} \\
        \hline
        1 Byte $= 8$ bit & $2^8 \text{Bytes} = 256 \text{Bytes}$ \\
        2 Bytes $= 16$ bit & $2^{16} \text{Byte} \cdot 2 \text{Bytes} = 256 \text{Megabytes}$ \\
        8 Bytes $= 64$ bit & $2^{64} \text{Byte} \cdot 8 \text{Bytes} = 512 \text{Terabytes}$ \\
        \hline
    \end{tabular}
\end{center}